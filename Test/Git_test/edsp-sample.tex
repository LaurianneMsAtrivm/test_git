%%
%% This is an example of a titlepage definition, which can be \included in
%% at the beginning of your document.
%%

\begin{edsp}
%% Nom de l'auteur
\auteur{}
test
%% Domaine : seule la premiere ligne non-commentee est prise en compte
%%           un avertissement est envoye si plusieurs domaines sont definis
%\domaine{Informatique}
%\domaine{Math\'ematiques -- Math\'ematiques appliqu\'ees}
%\domaine{Sciences physiques}
%\domaine{Chimie}
%\domaine{Sciences de la vie et de la sant\'e}
%\domaine{M\'ecanique -- G\'enie m\'ecanique -- G\'enie civil}
%\domaine{\'Electronique -- \'Electrotechnique -- Automatique}
%\domaine{Sciences \'economiques et de gestion}
%\domaine{Sciences sociales -- Philosophie -- Art}
%\domaine{Sciences de l'\'education -- Langues}

%% Sujet de la these
\sujet{Titre de ma th\`ese}

%% Date de soutenance (le package laissera un espace de 3cm si la date n'est pas definie)
%\date{}

%% ville de soutenance. Par defaut, c'est Cachan.
%\lieu{}

%% Numero d'ordre (donne par l'EDSP. Format YYYY/NNN)
%\numero{}

%% Jury (arguments obligatoires : Prenom Nom, titre, role (rapporteur, pr\'esident, ...)
%% exemple: \jury{John Doe}{Professeur}{Directeur de th\`ese}
%% Le jury est affiche dans l'ordre dans lequel il est defini
\jury{John Doe}{Professeur des universit\'es}{Directeur de th\`ese}
\jury{2}{}{}
\jury{3}{}{}
\jury{4}{}{}
\jury{5}{}{}
\jury{6}{}{}

%% Labo (adresse, logo)
% \labo{Nom de mon labo -- ENS~Cachan \\
%  61 avenue du pr\'esident Wilson --  F-94230 Cachan}
% \logolabo[height=10mm]{}
\end{edsp}
